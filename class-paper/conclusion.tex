Effective analysis of environmental justice concerns should be location-specific
and issue-focused to promote the most equitable access to incentive programs and
solutions. The argument advanced in this work is that prioritizing the distribution of solar
panels offered by programs like Solar for All or Illinois Shines will minimize the
overall heat-related adversity. Current standards for prioritizing beneficiaries
of these programs do not account for climate hazards such as heatwaves nor the
many vulnerabilities that, together, create a holistic understanding of risk.
Through this spatial analysis framework, we propose more targeted
analysis of priority beyond common methods of identifying environmental justice
communities for future allocation of solar incentive funding. This strategy could
be applied to urban environments to create a priority investment list to make
progress toward a more just energy transition nationwide.
