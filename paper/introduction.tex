Climate change will increase the frequency and severity of extreme heat events and,
due to the \ac{uhi} effect, urban centers are more susceptible to heat waves and
heat stress \cite{zhao_strong_2014,dahl_killer_2019}. The Union of Concerned
Scientists estimates that the number of days above 32 $^\circ$C in the Midwest
will increase five-fold by midcentury unless action is taken to reduce carbon
emissions and slow climate change  \cite{dahl_killer_2019}. The lack of strong
federal climate policies leaves individual states and institutions responsible
for acting on climate change. This case study investigates the optimal distribution
of rooftop solar panels that minimizes heat wave risk for the City of Chicago.

We chose Chicago as the focus of this work for two reasons. First, Chicago was the
epicenter of a deadly heat wave in 1995; one of the deadliest heat waves in United
States history. Over 700 people died in this heat wave\cite{klinenberg_heat_2003}.
Further, the death toll from this heat wave was exacerbated by a combination of
social factors including income and isolation. As a result of this heat wave,
Chicago officials recognized the need to prepare for future heat waves. This
work proposes rooftop solar panels as a possible mitigation strategy.
Second, the State of Illinois has programs such as Solar for All and Illinois
Shines that aim to provide greater access to clean energy for low-income populations
\cite{illinois_solar_for_all_environmental_2022}. Thus, the resources to increase
rooftop solar penetration already exist but lack guidance based on heat wave risk.

Illinois Solar for All and Illinois Shines are two programs strengthened by the
2021 \ac{ceja} \cite{harmon_climate_2021}. These policies intend to expand the
rooftop solar capacity in Illinois by providing incentives or subsidies for solar
installation, with specific allocations for low-income communities. Rooftop solar
panels help reduce the percentage of household income spent on energy costs,
known as the energy burden, through net-metering policies that pay consumers for
excess energy generation \cite{brown_high_2020}. This reduction of energy burden
improves resilience to heat waves by increasing access to air-conditioning for
low-income households during high-demand times when electricity is most expensive.
However, a high penetration of intermittent renewables, such as solar panels,
increases price volatility and the energy burden for consumers without solar
installations. This is called the ``paradox of renewable energy policy'' and
highlights the need for efficient prioritization of at-risk areas \cite{blazquez_renewable_2018}.
Therefore, the purpose of this study is to identify areas with the highest heat
wave risk so they may be prioritized by programs like Solar for All and Illinois
Shines.

In order to identify high-priority areas, we curated economic and demographic
data for Chicago, along with satellite data from the \ac{nsrdb} published by the
\ac{nrel}. We perform hierarchical clustering on this data to identify regions of
similar risk and suitability for solar panels. Section \ref{section:methods_data}
discusses details of the data selection and processing, section \ref{section:results}
presents the results of the clustering algorithm, and section \ref{section:discussion}
develops a descriptive typology based on the clustered areas.
