The prioritization mapping protocol outlined by this research is better suited for
identifying regions for targeted incentive distribution to address specific issues
in urban settings than existing general environmental justice community
identification tools currently used for funding dissemination.

\subsection{Comparison to other Energy Justice Mapping Tools}
Other screening tools have been developed to analyze spatial characteristics of
environmental justice. The \ac{ejscreen} tool from the \ac{epa} \cite{us_epa_ejscreen_2014}
and the US Council on Environmental Quality’s \ac{cejst} tool can both be used to
identify disadvantaged or vulnerable populations spatially
\cite{council_on_environmental_quality_climate_nodate}. However, the map generated
by this research identifies a more precise region of the city of Chicago to be
prioritized for rooftop solar initiatives, therefore informing more efficient
allocation of resources to the most vulnerable communities for this particular issue.
Figure \ref{fig:ejscreen_map} and Figure \ref{fig:cejst_map} show the at-risk areas
in Chicago identified by \ac{ejscreen} and \ac{cejst}, respectively.
\begin{figure}[H]
    \begin{center}
      \includegraphics[trim=0 200 0 120, clip, width=\columnwidth]{ejscreen_map}
      % \vspace*{-3cm}
      \caption{Environmental justice communities identified by \ac{ejscreen}. ``Yes''
      indicates that the community has environmental justice concerns.}
      \label{fig:ejscreen_map}
    \end{center}
\end{figure}

\begin{figure}[H]
    \begin{center}
      \includegraphics[trim=0 200 0 120, clip, width=\columnwidth]{cejst_map}
      % \vspace*{-3cm}
      \caption{Normalized energy burden from the \ac{cejst} tool.}
      \label{fig:cejst_map}
    \end{center}
\end{figure}

Importantly, neither of these tools incorporate heatwave risk and use a limited
set of social features. The prioritization created in this research considers
traditional environmental justice factors, including age, income, and race, but
also incorporates urban heat island impacts, qualified roof area, access to cooling
centers, tree cover, and energy burden to more specifically interrogate spatial
distribution of heat impacts that could be mitigated through access to cleaner,
cheaper energy for cooling. This is a more holistic approach to understanding
environmental justice for the specific issue of access to affordable energy to
cool urban homes in a heat wave. While mapping the distribution of spatial
attributes of environmental justice can illuminate inequities at a multitude of
scales, this more precise issue-specific mapping could be more effective for
policy and incentive development.

\subsection{Targeted Incentive Distribution}

Illinois \ac{sfa} is a state-funded program that “promotes equitable access to
the solar economy through program incentives that help make solar more affordable
for low-income communities” \cite{illinois_solar_for_all_environmental_2022}. The
program allows low-income communities to benefit from community solar arrays and
distributed generation installations, as well as providing low-cost solar
installations to non-profit organizations and public facilities. Initial funding
for SFA was provided by Illinois' 2017 Future Energy Jobs Act, but when funds were
exhausted, talk of a “solar cliff” highlighted the uncertainty of the future of
solar incentives in the state (Lydersen, 2020). The popularity of this program
and limited funding for solar projects emphasizes the need for targeted distribution
of funds to the communities that need it most.

SFA has an explicit commitment to increasing access to solar energy among
environmental justice communities. SFA uses \ac{epa}’s \ac{ejscreen} and a
self-designation  process to identify environmental justice communities across
the state of Illinois. However, this method of identifying communities to target
for SFA incentives is based on a more general understanding of environmental
justice, and does not necessarily consider the potential impact of solar energy
to reduce the energy burden on low-income households through solar installation.
Targeted prioritization as outlined in this research could result in allocation
of limited solar incentive funds to the communities that would experience the
greatest benefit from solar installations by reducing risks associated with heatwaves.
